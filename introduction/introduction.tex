\documentclass[../main.tex]{subfiles}
\graphicspath{{\currfiledir}}
\begin{document}
\chapter{Introduction}
The Maya civilization flourished in Mesoamerica from approximately 2000 BCE to 1500 CE 
(\cite[3]{estrada-belli2011}) and left behind a highly sophisticated system of hieroglyphic writing. 
This intricate writing system was used to inscribe symbols on a wide range of 
objects, including ceramics, sculptures, stairways, lintels, buildings, 
and monuments (\cite[\ppno~17--27]{thompson1962}). 
These hieroglyphs recorded diverse subjects, such as personal possessions and dedications, 
religious beliefs, political events, and historical occurrences.

In addition to the famous writings on stelae, temples and altars, the Maya also wrote on sheets of 
bark paper (\cite[34\psq]{vonhagen1944}) which were then bundled and 
folded to books (\Cref{fig:introduction-example-dresden-codex} for some sample pages).
These books are usually called \emph{codices}.
\begin{center}
    \includegraphics[width=\textwidth,keepaspectratio]{img/example-dresden-codex}
    \captionof{figure}{Sample pages of the \emph{Dresden Codex} unfolded}
    \label{fig:introduction-example-dresden-codex}
\end{center}
Unfortunately, most of the codices are lost now.
Only four books are known to have survived throughout the time, 
namely the \emph{Dresden Codex}, the \emph{Madrid Codex} (also called \emph{Troana Codex}),
the \emph{Paris Codex} and \emph{Maya Codex of Mexico} (also known as \emph{Grolier Codex}).
The codices are named after the location where they have been found or stored.
Hieroglyphs can also be found on incised bones, shells, Jade and Hard stones and on painted or 
carved pottery (\Cref{fig:introduction-panel-with-royal-woman} and
\Cref{fig:introduction-vessel-with-battle-scene}).
\begin{figure}
    \centering
    \subfloat[][]{
        \centering
        \includegraphics[width=0.47\textwidth]{img/panel-with-royal-woman}
        \label{fig:introduction-panel-with-royal-woman}
    }
    \hfill
    \subfloat[][]{
        \centering
        \includegraphics[width=0.47\textwidth]{img/vessel-with-battle-scene}
        \label{fig:introduction-vessel-with-battle-scene}
    }
    \caption[Maya art from the Classic and Late Classic]{Maya art from the Classic and Late Classic.
             \subref{fig:introduction-panel-with-royal-woman} Panel with royal woman, 
             Classic Period, Cleveland Museum of Art, Purchase from the J. H. Wade Fund 1962.32;
             \subref{fig:introduction-vessel-with-battle-scene} Vessel with Battle Scene, 
             Late Classic Period, Cleveland Museum of Art, John L. Severance Fund 2012.32}
\end{figure}

Besides hieroglyphic texts, several manuscripts written in Yucatec Maya using the Latin alphabet 
have been written  during 16th, 17th and 18th century.
\textcquote[68]{brinton1882}{The ``old writings'' \elide were composed by natives 
who had learned to write the Maya in the alphabet adopted by the early missionaries and 
conquerors\elide There were at one time numerous of these records.}
Many Maya villages had such a book.
All books claim to be written by \emph{Chilam Balam} --- a great prophet from the 16th century
who predicted the arrival of the Spaniards in Yucat\'{a}n
(\textcquote[\ppno~186--187]{roys1933}{\elide bearded man would come from the east and 
introduce a new religion}).
Every book is named after the town the manuscript was located in, e.g. 
\emph{The Book of Chilam Balam of Tizimin} originated from \emph{Tizimin} --- a town 
in Yucat\'{a}n
(\Cref{fig:introduction-chilam-balam-of-tizimin-cover-page} shows the title page).
Some other books still exist, many have been lost during the colonial period.
\begin{figure}[ht]
    \centering
    \subfloat[][]{
        \centering
        \includegraphics[width=0.47\textwidth]{img/chilam-balam-of-tizimin-cover-page}
        \label{fig:introduction-chilam-balam-of-tizimin-cover-page}
    }
    \hfill
    \subfloat[][]{
        \centering
        \includegraphics[width=0.47\textwidth]{img/chilam-balam-of-ixil-page-20r}
        \label{fig:introduction-chilam-balam-of-ixil-page-20r}
    }
    \caption[Sample pages of the books of \emph{Chilam Balam}]{Sample pages of the books of 
             \emph{Chilam Balam}.
             \subref{fig:introduction-chilam-balam-of-tizimin-cover-page} Cover of the 
             \emph{Book of Chilam Balam of Tizimin} (\emph{Chilam Balam ``C\'odice Tizimin''});
             \subref{fig:introduction-chilam-balam-of-ixil-page-20r} Page 20r of 
             \emph{Book of Chilam Balam of Ixil}. It shows the \haab months and \tzolkin days.}
\end{figure}
These books also play an important role during the process of hieroglyphic decipherment as they 
contain vital information about historical events, calendrical information, ritual and 
medical descriptions and many more aspects which help to analyze the great writings of the 
Maya (\cite[3\psq]{roys1933}).
For instance, some pages of these books for instance, shows some 
calendrical information from the \tzolkin and \haab calendars 
(\Cref{fig:introduction-chilam-balam-of-ixil-page-20r}).

\section{Early attempts in deciphering the Maya hieroglyphs}
For centuries, Maya hieroglyphs remained a mystery, as the script was undeciphered and the language 
spoken by the ancient Maya was largely unknown. 
It wasn’t until the late 19th and early 20th centuries that scholars made significant progress in 
unlocking the secrets of the hieroglyphs, beginning to understand their grammar and syntax.

One key breakthrough came in the 1960s when Yuri Knorosov (\cite{knorozov1967}) demonstrated that 
the script was a fully developed writing system, combining both phonetic and logographic elements. 
While his work is often cited, it built upon earlier efforts and laid the groundwork for further 
advances in decipherment. 
His discovery enabled scholars to approach Maya texts with greater rigor and set the 
stage for continued research into their language and culture.

The decipherment of Maya hieroglyphs owes its success to the tireless work of epigraphers, 
linguists, and archaeologists who dedicated their lives to studying the Maya civilization and 
its written language. 
Their efforts have illuminated our understanding of this ancient culture and its achievements.

Today, the field remains active, with new discoveries about the Maya language and culture 
being made regularly (\cite{zender2017}). 
The decipherment of Maya hieroglyphs has revealed remarkable insights 
into the history and culture of the ancient Maya.
There are still many questions and not all signs can be read, and even of those which can be read
many signs are still not fully understood.
Yet, it is fascinating that it was possible to decipher most parts
and the nature of the script up to a degree so that scholars are able to read the writings and 
understand the texts on the monuments and objects.

This work explores the history of Maya hieroglyph decipherment, from early attempts to understand 
the symbols to modern techniques used today. 
Along the way, it highlights the key figures who contributed significantly to this 
field and the impact their work has had on our understanding of the Maya civilization. 
As an ongoing process, it is not a finalized document but evolves over 
time, revealing more about the ancient script with each new discovery.

\section{Outlook}
This book invites readers to embark on a journey through the fascinating process of 
deciphering Maya hieroglyphs. 
It seeks to answer questions such as: 
\enquote{How did scholars determine the meaning of sign X?} 
\enquote{What methods and scientific approaches have been used to tackle an almost unknown script?} 
These are just a few of the many inquiries this work aims to address.

The narrative begins in the past, around 1830, with the discovery of an old manuscript believed 
to have been written by Bishop Diego de Landa in the 16th century. 
From there, it traces discoveries made during the 19th and 20th centuries, ultimately bringing 
readers up to date with modern advancements.

As the book progresses, deeper insights into the Maya script will be revealed. 
The chapters are organized chronologically, providing a 
clear explanation of the progress and discoveries made by scholars over time. 
While the story is primarily told in chronological order, 
occasional jumps in time may occur to enhance understanding of specific aspects of 
the decipherment process.

Each scientific work mentioned will include citations and references to help readers access 
additional literature if they choose. 
Although not every decipherment can be detailed here, the provided sources are intended to guide 
readers toward further exploration.

This book is designed to offer as many details as possible to ensure that readers understand 
the reasoning behind each step of the 
decipherment process, making the narrative both comprehensive and coherent.

\end{document}
